\documentclass[11]{article}
\usepackage{graphicx}
\graphicspath{ {images/} }


\title{CS2006 Haskell Project 2 - \\ Gomoku}
\date{27/04/2018}
\author{Matriculation Numbers: 160001362, 160016245, 160021429 (Group 14)}

\begin{document}
	\maketitle
	\newpage
	\tableofcontents
	
	\newpage
	\section{Summary of Functionality}
		This practical specified the development of the board game Gomoku using the functional programming language Haskell. \\\\The provided README file gives detailed instructions explaining how to execute the program including configuring settings both through the terminal as well as in game.
\\\\The following functionality has been implemented:
	\subsection{Basic Specification:}
		All requirements from the basic specification have been implemented. They are as follows:	
		\begin{enumerate}
			\item \textbf{Implement the game mechanics in Board.hs} -
			\item \textbf{Implement the drawWorld function in Draw.hs to display the current board state graphically} -
			\item \textbf{Implement appropriate event handlers for inputs events} -
			\item \textbf{Implement a move generator (in AI.hs) and an evaluation function (in Board.hs) to provide a computer opponent} - two AIs have been implemented, a random AI and a heuristic based AI. The random AI generates a random index to refer to an element in a list of empty positions of the board and places a piece there. The heuristic AI uses the evaluation function in Board.hs to calculate its board score relative to its opponent for each possible move and chooses the highest scoring move.
		\end{enumerate}
	
	\subsection{Additional Requirements:}
	 From the suggested additional requirements, all of the easy and medium requirements have been implemented, as have the listed hard requirements:
	 	\subsubsection{Easy}
			\begin{itemize}
				\item \textbf{Easy Requirement 1} - 
				\item \textbf{Easy Requirement 2} - 
				\item \textbf{Easy Requirement 3} - 
			\end{itemize}

		\subsubsection{Medium}
			\begin{itemize}
				\item \textbf{Medium Requirement 1} - 
				\item \textbf{Medium Requirement 2} - 
				\item \textbf{Medium Requirement 3} - 
				\item \textbf{Medium Requirement 4} -
			\end{itemize}
		\subsubsection{Hard}
				\begin{itemize}
					\item \textbf{Hard Requirement 2} - 
				\end{itemize}
	\subsection{Further Features:}
		The following additional features have also been implemented:
		\begin{itemize}
			\item 
		\end{itemize}

	\section{Design and Implementation}
		\begin{itemize}
				\item
		\end{itemize}
		
	\section{Evidence of Testing}	
		
		
	\section{Known Problems}
		
		
	\section{Problems Overcome}
	

	\section{Summary of Provenance}
			\begin{itemize}
				\item 160001362:
					\begin{itemize}
						\item Basic Requirement 4
						\item Medium Requirement 3 - Implemented the evaluation for the AI which is also used to generate a user hint.
						\item Hard Requirement 2 - Implemented the multiple AIs
					\end{itemize}
					
				\item 160016245:
					\begin{itemize}
						\item Easy Requirement 1
						\item Easy Requirement 3 - Rule of four and four
						\item Medium Requirement 1
						\item Medium Requirement 3 - Displayed the hints generated by the evaluation function.
					\end{itemize}
					
				\item 160021429:
					\begin{itemize}
						\item Basic Requirements 1 - 3
						\item Easy Requirement 2
						\item Medium Requirement 2
						\item Medium Requirement 4
					\end{itemize}
			\end{itemize}
				
	
\section{Conclusion}


\end{document}